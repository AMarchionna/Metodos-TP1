\documentclass[10pt,a4paper]{article}
\usepackage[utf8]{inputenc} % para poder usar tildes en archivos UTF-8
\usepackage[spanish]{babel} % para que comandos como \today den el resultado en castellano
\usepackage{a4wide} % márgenes un poco más anchos que lo usual
\usepackage{caratula}
\usepackage{verbatimbox}
%\usepackage{algorithm}
%\usepackage{algorithmic}
%\usepackage{algpseudocode}
\usepackage{float} %para decir a donde van los pseudocodigos
\usepackage{icomma}
\usepackage{multicol}
\usepackage{todonotes}
\usepackage[nottoc]{tocbibind}
\usepackage{amsthm}
\usepackage{amssymb} %Simbolos matematicos
\usepackage{amsmath}
\usepackage{mathtools}
\usepackage{tikz,pgf}
\usetikzlibrary{babel,arrows,automata,graphs,graphs.standard,quotes,decorations.markings} %Para hacer grafos con tikz
\usepackage{multirow} %Para las tablas

\usepackage{rotating} %Para escribir vertical


%\alglanguage{pseudocode}


\newcommand{\bigO}{\mathcal{O}}

\newcommand{\grafiquito}[2]{%
    \begin{figure}[H]
    \centering
    \includegraphics[width=7cm]{./#1.png}
    \caption{#2}
    \label{fig:#1}
    \end{figure}%
}

\newcommand{\png}[2]{%
    \begin{figure}[H]
    \centering
    \includegraphics[width=\textwidth]{./img/#1.png}
    \caption{#2}
    \label{png:#1}
    \end{figure}%
}

\newcommand{\jpg}[2]{%
    \begin{figure}[H]
    \centering
    \includegraphics[width=\textwidth]{./img/#1.jpg}
    \caption{#2}
    \label{jpg:#1}
    \end{figure}%
}

\tikzset{
  big arrow/.style={
    decoration={markings,mark=at position 1 with {\arrow[scale=4,#1]{>}}},
    postaction={decorate},
    shorten >=0.4pt},
  big arrow/.default=black}
  
\newtheorem{obs}{Observaci\'on}[section]
\newtheorem{lema}{Lema}[section]
\newtheorem{ejemplo}{Ejemplo}[section]
\renewcommand*{\proofname}{Demostraci\'on}


\setlength{\parindent}{2em}
\setlength{\parskip}{1em}

\begin{document}

\titulo{Trabajo Práctico 1}
\subtitulo{Sistemas de ecuaciones lineales}

\fecha{\today}

\materia{Métodos Numéricos}
\grupo{}

\integrante{}{}{}
\integrante{}{}{}
\integrante{Marchionna, Agustín Luis}{823/17}{agusmarchionna1998@gmail.com}
\integrante{Cribioli, Ezequiel}{}{}

\maketitle


\newpage
\begin{center}
\end{center}
\thispagestyle{empty}
\setcounter{page}{0}

\newpage
\tableofcontents

% \twocolumn

\newpage
\section{Introducción}

En la actualidad existen diversas técnicas para determinar la tabla de posiciones en una competencia compuesta por varios equipos. En este trabajo vamos a estudiar varias de ellas, principalmente el método de la Mátriz de Colley (CMM), elaborado por Wes Colley.

Primero, debemos entender cuál es el problema a resolver.

\noindent \textbf{Ranking de equipos} En una competencia, compuesta por $T$ equipos, se enfrentan en cada uno de $P$ encuentros $2$ de los equipos. En cada partido, exactamente uno de los $2$ resulta ganador, por lo que no hay empates. El objetivo es determinar un orden entre los equipos, que refleje el desempeño de los mismos.

Para entender los distintos métodos posibles, veamos algunos ejemplos.

\begin{ejemplo}
	\normalfont
	Supongamos que en una competencia hay $6$ equipos, y que se han jugado $10$ partidos, en donde cada equipo anota goles y el ganador del partido resulta aquel que más goles convirtió, y no es posible el empate. Los resultados se pueden observar en la siguiente tabla:

	\begin{table}[h]
		\centering
		\begin{tabular}{|c|c|c|c|}
			\hline
			\textbf{Equipo local} & \textbf{Goles del local} & \textbf{Equipo visitante} & \textbf{Goles del visitante} \\ \hline
			1                     & 16                       & 4                         & 13                           \\ \hline
			2                     & 38                       & 5                         & 17                           \\ \hline
			2                     & 28                       & 6                         & 23                           \\ \hline
			3                     & 34                       & 1                         & 21                           \\ \hline
			3                     & 23                       & 4                         & 10                           \\ \hline
			4                     & 31                       & 1                         & 6                            \\ \hline
			5                     & 33                       & 6                         & 25                           \\ \hline
			5                     & 38                       & 4                         & 23                           \\ \hline
			6                     & 27                       & 2                         & 6                            \\ \hline
			6                     & 20                       & 5                         & 12                           \\ \hline
		\end{tabular}
	\end{table}
	
	
\end{ejemplo}
 

\newpage
\input{Secciones/Resolucion.tex}

\newpage
\input{Secciones/Experimentacion.tex}

\newpage
\input{Secciones/Conclusiones.tex}

\newpage

\end{document}